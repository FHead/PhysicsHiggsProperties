\documentclass[11pt,t]{beamer}
\usepackage[latin1]{inputenc}
\usepackage{amsmath}
\usepackage{amsfonts}
\usepackage{amssymb}
\usepackage{graphicx}
\usepackage{tikz}
\usetheme{default}
\setbeamertemplate{footline}[frame number]

\title{Status Update on Measuring Higgs anomalous HVV couplings with 8D maximum likelihood fit}
\subtitle{Status and plans - draft(v2) for Wednesday}
\author{Yi Chen + a lot of people to be included}
\institute{California Institute of Technology}
\date{\today}

\begin{document}

\begin{frame}[plain]
   \titlepage
\end{frame}

\begin{frame}{Table of contents}
   \begin{itemize}
   \item Status
   \item Quick overview of strategy and convolution
   \item Some words on bug fixes and improvements
   \item Near-term plans
   \item Summary
   \end{itemize}
\end{frame}

\begin{frame}{Status}
   \begin{itemize}
   \item Many rounds of bug fixing in the past couple of months - all known bugs/features fixed.
   More details in later slides.
   \item Rethink many steps to make sure our result is robust and easier to obtain
      \begin{itemize}
      \item For example the normalization of one set of pdf can be achieved to O(0.1\%) in 10 minutes
      as opposed to hours or even days
      \end{itemize}
   \item Once we are happy with the analysis, carry out the measurement on real data
   will take half a day from beginning to end, reproducing all numbers from scratch.
   \end{itemize}
\end{frame}

\begin{frame}{Quick overview of strategy}
   \begin{itemize}
   \item Construct likelihood as a function of Lagrangian parameters
      \begin{itemize}
      \item Obtain differential cross sections at generator level
      \item Perform an integral event by event convoluting generator level distribution
      with parameterization of detector response for leptons
      \item Perform a calculation of the pdf normalization
      \end{itemize}
   \item Perform maximization on data to extract parameters of interest
   \item Current Lagrangians that we have differential cross sections calculated:
   \begin{eqnarray}
   L \simeq M_Z^{-2} A_1^{ZZ} X Z_\mu Z^\mu + A_2^{ZZ} X Z_{\mu\nu} Z^{\mu\nu} + A_3^{ZZ} X Z_{\mu\nu} \tilde{Z}^{\mu\nu} + ...\nonumber\\
   + A_2^{ZA} X Z_{\mu\nu} F^{\mu\nu} + A_3^{ZA} X Z_{\mu\nu} \tilde{F}^{\mu\nu}\nonumber + ...\nonumber\\
   + A_2^{AA} X F_{\mu\nu} F^{\mu\nu} + A_3^{AA} X F_{\mu\nu} \tilde{F}^{\mu\nu}\nonumber + ...\nonumber
   \end{eqnarray}
   \item $Z \rightarrow 4l$ coming soon.  Will be important cross check to measure Z spin.
   \end{itemize}
\end{frame}

\begin{frame}{Quick overview of convolution}
   \begin{itemize}
   \item Reconstructed level likelihood is the integral of all generator level configuration times the probability
   that the generator level configuration is smeared into reconstructed level configuration
   \begin{equation}
      P^R(\vec{X} | \vec{A}) = \int P^G(\vec{X}' | \vec{A}) T(\vec{X} | \vec{X}') d \vec{X}'\nonumber
   \end{equation}
   \item $\vec{X}$ is the vector of all kinematic variables: $M_H, p_T^H, \phi_H, y_H, \phi_{offset}$, $\Phi$, $\Phi_1$, $\theta_1, \theta_2, \Theta$, $M_1, M_2$
   \item $P^G(\vec{X}')$ is the generator level differential cross section
   \item $T(\vec{X} | \vec{X}')$ is the probability density that configuration $\vec{X}$ appears given generator level configuration $\vec{X}'$
   \end{itemize}
\end{frame}

\begin{frame}{Some discussions}
   \begin{itemize}
   \item There are a lot of Jacobian factors to be taken care of
   \item Narrow width approximation inserts a delta function in $M_H^2$, which is non-trivial to take care of in the convolution integral
   \item Considerable amount of time was spent making sure all the factors are correct and validated, which was buggy before...
   \item There are near-cancellation of terms which won't show up easily in many places except in final fit biases.
   Projections will look fine, but bias show up in the fits.
   A stronger test with likelihood distribution is successful in catching potential bugs.
   \item Dimensionality isn't the point as long as consistent likelihood is used in a correct way.
   If we choose generator likelihood ratio as the variable to use, it's certainly less optimal
   compared to the exact same analysis using reconstructed level likelihood ratio.
   \textbf{Will this piss people off?}
   \end{itemize}
\end{frame}

\begin{frame}{Some discussion on strategy}
   \begin{itemize}
   \item Production variables enter through detector acceptance effects.  To reduce dependence on
   production variables, we average out the 4 dimensions irrelevant to $HVV$ vertex ($\phi_H, p_T^H, \phi_{offset}, y_H$)
   \item Systematics can be easily incorporated with nuisance parameters
   \item This approach is the simplest conceptually
   \item We fit for ratios/fractions and not worry about absolute rates
   \item Can fit simultaneously for all possible parameters as long as we have enough events
   \item Deal with fake background by making a big template + large systematics to cover possible variations.
   \item \textbf{Whatever else we want to talk about here...}
   \end{itemize}
\end{frame}

\begin{frame}{Validation of convolution}
   \begin{itemize}
   \item An example projection comparison plots are shown here; more (and more stats!) will be shown when we
   present details on the convolution integral.
   \item This is the reconstructed Higgs mass distribution for standard model signal.
   At generator level this is a delta function, all the non-trivial distribution comes from smearing.
   \begin{figure}
   \includegraphics[width=0.5\textwidth]{SignalZZMass}
   \end{figure}
   \item A stronger test on likelihood distribution is presented in the next slide.
   \end{itemize}
\end{frame}

\begin{frame}{Validation of convolution}
   \begin{itemize}
   \item Distribution of likelihoods must match between events generated by the pdf (RECO level)
   and Madgraph samples after detector effects
   \item This is a stronger check than individual projections
   \end{itemize}
   \begin{figure}
   \includegraphics[width=0.5\textwidth]{BackgroundLikelihoodAfterEFix.pdf}
   \includegraphics[width=0.5\textwidth]{SignalLikelihoodAfterEFix.pdf}
   \caption{Likelihood comparison between generated samples and Madgraph samples at RECO level.
   Left: Background, Right: SM signal}
   \end{figure}
\end{frame}

\begin{frame}{Timeline and plans}
   Statistics on toy events are accumulating right now...  Plan to finish these basic analysis parts as soon as possible.
   Result with lower stats look reasonable.
   \begin{itemize}
   \item Bias studies - whether we can extract things correctly (with pull distributions to show that errors are reasonable)
   \item Projection to different amount of statistics
   \end{itemize}
   Present details bit by bit for scrutiny in this group from now on, hopefully we can open the box in a few months.

   Example pull distributions on some of the ratios of Lagrangian parameters is shown: better plots will come in later presentations
   \begin{figure}
   \includegraphics[width=0.3\textwidth]{FirstRecoResult_100+25_10_10.pdf}
   \includegraphics[width=0.3\textwidth]{FirstRecoResult_100+25_12_12.pdf}
   \includegraphics[width=0.3\textwidth]{FirstRecoResult_100+25_14_14.pdf}
   \end{figure}
\end{frame}

\begin{frame}{Summary}
   \begin{itemize}
   \item Rounds of bug fixes done.  Finally ready to move on again.
   \item Various improvements have been implemented that speed things up and make things more robust
   \item Toy results will appear soon; and in the meantime the analysis will be presented in pieces for scrutiny.
   \item We are preparing a document on all the details of calculation.
   \end{itemize}
\end{frame}

\begin{frame}{Backup slides ahead}
\end{frame}


\end{document}




